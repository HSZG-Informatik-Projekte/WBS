\documentclass[10pt,fleqn]{beamer}

\usepackage{pgfpages}

\setbeameroption{show notes}

\usepackage{multimedia}
\usepackage[normal,tabtopcap,figbotcap]{subfigure}
\usepackage{cite}

\setlength{\fboxsep}{0.0pt}
\setlength\fboxrule{0.5pt}
\usepackage{tikz}

\usepackage[ngerman]{babel}
\usepackage[utf8]{inputenc}
\usepackage{blindtext}
\usepackage{url}
\usepackage{wrapfig}
\usepackage{graphicx}


\title[WBS]{Williams-Beuren-Syndrom App}
\subtitle[]{Software Engineering II}
%\date{\today}
\date{26. Juni 2019}
\author{Sebastian Thomas, Leander Frieske, Toni Grüning}
\logo{\includegraphics[scale=.25]{Bilder/banner_zigr_gross.png}}
\usetheme{Madrid}

%\AtBeginSection[]{\begin{frame}\begin{block}{}\begin{center}\vspace{0.5cm}	\Large\textcolor{blue}{\insertsection}\vspace{0.5cm}\end{center}\end{block}\end{frame}}


\begin{document}
%\maketitle
%\setcounter{tocdepth}{1}

\begin{frame}
	\titlepage
\end{frame}

\begin{frame}{Gliederung}
 	\begin{columns}
   		\begin{column}{10.0cm}
		    \renewcommand{\baselinestretch}{1.5}
		    \normalsize
		    \tableofcontents
		    \renewcommand{\baselinestretch}{1.0}
		    \normalsize
	   	\end{column}
	   	\hspace*{-1.3cm}
	\end{columns}
\end{frame}


\section{Ausgangssituation}
\begin{frame}[t]{Ausgangssituation}
	\begin{itemize}
		\item Analysedokument aus SE1
		\begin{itemize}
			\item Überblick zu WBS
			\item erste Vorstellungen zur App
			\item mehrere UseCases
		\end{itemize}
	\item Endgültiges Ziel $\rightarrow$ WBS-App(Wissen, Begeistern, Spielen)
		\begin{itemize}
			\item Kindern spielerisch den Umgang mit anderen Menschen und bestimmten Situationen beibringen
			\item Hilfe für Alltagssituationen schaffen
		\end{itemize}
	\item Idee
		\begin{itemize}
			\item durch Videos Situationen beschreiben
			\item[] $\rightarrow$ danach Frage zur Situation
			\item soll Verhaltensweisen festigen
		\end{itemize}
	\end{itemize}
\end{frame}


\section{WBS}
\begin{frame}[t]{WBS Williams-Beuren-Syndrom}
	\begin{itemize}
		\item seltene, genetisch bedingte Entwicklungsbeeinträchtigung
		\item in der Regel mit einer geistiger Behinderung
		\item Zugrunde liegt ein zufällig auftretender Genverlust auf Chromosom 7\footnote{Ewart et al., 1993} 
		\item Die Prävalenz liegt schätzungsweise bei 1:7.500\footnote{Stromme \& Bjomstad, 2002.} $->$ seltenen Erkrankung\footnote{wenn nicht mehr als 5 von 10.000 Menschen in der EU von ihr betroffen sind}
		\item Verknüpfungen im Gehirn der visuellen-, räumlichen- und motorischen Bereiche geschädigt
		\begin{itemize}
			\item verkürzten Aufmerksamkeitsspanne
			\item kognitiven Schwierigkeiten
			\item Störung in der visuellen Wahrnehmung
			\item Entwicklungsstörungen im fein- und grobmotorischen Bereich
		\end{itemize}
		\item Vielzahl von spezifischen Stärken
		\begin{itemize}
			\item ausgeprägtes Neugierverhalten
			\item gutes Gedächtnis
			\item hoher IQ
			\begin{itemize}
				\item Bereich der Musik und des Rhythmus
				\item großen Enthusiasmus Kontakt mit neuen Menschen aufzunehmen
			\end{itemize} 
		\end{itemize} 
	\end{itemize}
\end{frame}
\section{Projektgruppe}
	\begin{frame}[t]{Projektgruppe}
		\begin{itemize}
			\item Prof. Dr. Ingolf Prosetzky
			\begin{itemize}
				\item Leitender Professor der Forschungsgruppe zum Williams-Beuren-Syndrom der Hochschule Zittau/Görlitz
			\end{itemize}
			\item Thomas Dubiel
			\begin{itemize}
				\item Vater eines 5-Jährigen Jungen mit WBS
				\item Gründer www.kruemelkiste-app.de\footnote{Netzwerk aus Personen mit WBS}
				\item Entwickler
			\end{itemize}
			\item Toni Grüning (Projektleiter)
			\item Leander Frieske
			\item Sebastian Thomas
		\end{itemize}
	\end{frame}

	\begin{frame}[t]{Meetings}
		\begin{itemize}
			\item Kickoff der gesamten Projektgruppe
			\begin{itemize}
				\item Kennenlernen
				\item Vorstellung Projekt
				\item Festlegung Ziel
			\end{itemize}
			\item Statusbericht des Fortschritts
			\item Abschlussvorstellung im Rahmen der SE-Vorlesung 
		\end{itemize}
	\end{frame}



\section{Unsere Zielstellung}
\begin{frame}[t]{Unsere Zielstellung}
	\begin{itemize}
		\item eine funktionierende App
		\item grober Ablauf aus dem Analyseprojekt
		\item Grundgerüst für weiterführende Arbeiten an der Anwendung
		\item leichte Erweiterbarkeit sowie Anpassbarkeit
		\item einfache Bedienung der App
	\end{itemize}
\end{frame}


\section{Einfache App}
\begin{frame}[t]{Einfache App}
	\begin{itemize}
		\item Technische Umsetzung
		\subsection{Umsetzung}
		\begin{itemize}
			\item Anwendung: Android App
			\item IDE (Entwicklungsumgebung): Android Studio
			\item Programmiersprache: Java
			\item Layout: XML
			\item Persistenz (Datenspeicherung): JSON
			\item Versionierung/Verwaltung: GitLab/GitHub
		\end{itemize}
		\item Anforderungsumsetzung
			\begin{itemize}
			\item einfache Erweiterbarkeit durch JSON-Format
			\item[] $\rightarrow$ textuelles Eintragen neuer Ressourcen
			\end{itemize}
		\item Herausforderungen
		\subsection{Herausforderungen}
		\begin{itemize}
			\item für uns neue Entwicklungsumgebung
			\item neue Zielplattform und damit neue Art z.B. Bilder einzubinden
			\item grafisch ansprechende Oberfläche gestalten
		\end{itemize}
	\end{itemize}
\end{frame}

\section{Live Demonstration}
\begin{frame}[c]{Demo}
	\begin{block}{}
		\begin{center}
			\begin{Huge}
			Live Demonstration
			\end{Huge}
		\end{center}
	\end{block}
\note{\begin{itemize}
	\item starten der App $\rightarrow$ Profil anlegen
	\item schließen und wieder starten
	\item Profil ansehen $\rightarrow$ verändern
	\item Video ansehen $\rightarrow$ Frage beantworten erst falsch dann richtig
	\item Profil $\rightarrow$ Stern
	\item Button nach Amerika $\rightarrow$ gleiche Fragen schon ausgefüllt
	\item Profil löschen $\rightarrow$ Belohnungen weg sowie Progress
	\item fertig
	\item eventuell etwas einbinden im Code zeigen?
\end{itemize}}
\end{frame}

\section{weitere Möglichkeiten}
\begin{frame}[t]{weitere Möglichkeiten}
	\begin{itemize}
		\item Fachlich
		\begin{itemize}
			\item fachgerechte Videos und entsprechende Fragen bzw. Antworten
			\item wo das Kind z.B. eine To-Do Liste abrufen kann, was es machen soll,
			wenn z.B. der Bus nicht kommt oder welchen Weg muss ich von der Schule bis nach
			Hause gehen?
			\item Shop, in dem man Erweiterungen gegen Sterne eintauschen kann
		\end{itemize}
		\item Technisch
		\begin{itemize}
			\item Stabilität verbessern $\rightarrow$ Fehlerfangen, Bugs beseitigen
			\item Bereitstellung für andere OS
			\item Support für Gerätegrößen wie Tablet
			\item Querformat
		\end{itemize}
		\item Grafik/Audio
		\begin{itemize}
			\item Einbau des Begleiters als Figur
			\item Überarbeitung der auf der Oberfläche befindlichen Bilder und des Pfades
			\item Animieren z.B. des Begleiters
			\item anregendes Aussehen um Interesse aufrecht zu erhalten
			\item musikalische Untermalung
		\end{itemize}
	\end{itemize}
\end{frame}

\end{document}


