\documentclass[10pt,fleqn]{beamer}

\usepackage{multimedia}
\usepackage[normal,tabtopcap,figbotcap]{subfigure}
\usepackage{cite}

\setlength{\fboxsep}{0.0pt}
\setlength\fboxrule{0.5pt}
\usepackage{tikz}

\usepackage[ngerman]{babel}
\usepackage[utf8]{inputenc}
\usepackage{blindtext}
\usepackage{url}
\usepackage{wrapfig}
\usepackage{graphicx}


\title[WBS]{Williams-Beuren-Syndrom App}
\subtitle[]{Software Engineering II}
\date{\today}
%\date{26. Oktober 2018}
\author{Sebastian Thomas, Leander Frieske, Toni Grüning}
\logo{\includegraphics[scale=.25]{Bilder/banner_zigr_gross.png}}
\usetheme{Madrid}

%\AtBeginSection[]{\begin{frame}\begin{block}{}\begin{center}\vspace{0.5cm}	\Large\textcolor{blue}{\insertsection}\vspace{0.5cm}\end{center}\end{block}\end{frame}}


\begin{document}
%\maketitle
%\setcounter{tocdepth}{1}

\begin{frame}
	\titlepage
\end{frame}

\begin{frame}{Gliederung}
 	\begin{columns}
   		\begin{column}{10.0cm}
		    \renewcommand{\baselinestretch}{1.5}
		    \normalsize
		    \tableofcontents
		    \renewcommand{\baselinestretch}{1.0}
		    \normalsize
	   	\end{column}
	   	\hspace*{-1.3cm}
	\end{columns}
\end{frame}


\section{Ausgangssituation}
	\begin{frame}[t]{Ausgangssituation}
		\begin{itemize}
			\item Analysedokument aus SE1
			\begin{itemize}
				\item Überblick zu WBS
				\item erste Vorstellungen zur App
				\item mehrere UseCases
			\end{itemize}
		\end{itemize}
	\end{frame}

\subsection{Analysedokument SE 1}
\begin{frame}[t]{Projektgruppe}
„WBS-App“ – Wissen, Begeistern, Spielen

mehrere Uses Cases

Beschreibung der Idee

WBS-Betroffenen fällt es schwer, verschiedene Situationen im Alltag zu meistern,
wie z.B. den emotionalen Umgang mit Menschen.

WBS-Lern App, mit dem
Schwerpunkt des emotionalen Umgangs mit Menschen.

Kinder spielerisch den Umgang mit verschiedenen alltäglichen Situationen im Umgang
mit Menschen erlernen.
\end{frame} 


\section{Projektgruppe}
	\begin{frame}[t]{Projektgruppe}
		\begin{itemize}
			\item Prof. Dr. Ingolf Prosetzky
			\begin{itemize}
				\item Leitender Professor der Forschungsgruppe zum Williams-Beuren-Syndrom der Hochschule Zittau/Görlitz
			\end{itemize}
			\item Thomas Dubiel
			\begin{itemize}
				\item Vater eines 5-Jährigen Jungen mit WBS
				\item Gründer www.kruemelkiste-app.de\footnote{Netzwerk aus Personen mit WBS}
				\item Entwickler
			\end{itemize}
			\item Toni Grüning (Projektleiter)
			\item Leander Frieske
			\item Sebastian Thomas
		\end{itemize}
	\end{frame}

\subsection{Projektgruppe}
\begin{frame}[t]{Meetings}
\begin{itemize}
	\item Kickoff der gesamten Projektgruppe
	\begin{itemize}
		\item Kennenlernen
		\item Vorstellung Projekt
		\item Festlegung Ziel
	\end{itemize}
	\item Vorstellen des Programmierten
	\item Abschlussvorstellung im Rahmen der SE-Vorlesung 
\end{itemize}
\end{frame}

\subsection{WBS}
	\begin{frame}[t]{WBS Williams-Beuren-Syndrom}
		\begin{itemize}
			\item seltene, genetisch bedingte Entwicklungsbeeinträchtigung
			\item in der Regel mit einer geistiger Behinderung
			\item Zugrunde liegt ein zufällig auftretender Genverlust auf Chromosom 7\footnote{Ewart et al., 1993} 
			\item Die Prävalenz liegt schätzungsweise bei 1:7.500\footnote{Stromme \& Bjomstad, 2002.}
			\item seltenen Erkrankung
			\item Verknüpfungen im Gehirn der visuellen-, räumlichen- und motorischen Bereiche geschädigt
			\begin{itemize}
				\item verkürzten Aufmerksamkeitsspanne
				\item kognitiven Schwierigkeiten
				\item Störung in der visuellen Wahrnehmung
				\item Entwicklungsstörungen im fein- und grobmotorischen Bereich
			\end{itemize}
			\item Vielzahl von spezifischen Stärken,
			\begin{itemize}
				\item ausgeprägtes Neugierverhalten,
				\item gutes Gedächtnis
				\item hoher IQ
				\begin{itemize}
					\item Bereich der Musik und des Rhythmus
					\item großen Enthusiasmus Kontakt mit neuen Menschen aufzunehmen
				\end{itemize} 
			\end{itemize} 
		\end{itemize}
	\end{frame}

\subsection{Kickoff}

\section{Zielstellung}
\begin{frame}[t]{Zielstellung}
	\begin{itemize}
		\item eine Funktionierende App
		\item grober Ablauf aus dem Analyseprojekt
		\item leichte Erweiterbarkeit sowie Anpassbarkeit
		\item einfache Bedienung der App
	\end{itemize}
\end{frame}


\section{Einfache App}
\begin{frame}[t]{Einfache App}
	\begin{itemize}
		\item Beschleunigung von Programmen durch Parallelisierung
	\end{itemize}
\end{frame}

\subsection{Technische Umsetzung}
\begin{frame}[t]{Technische Umsetzung}
	\begin{itemize}
		\item Anwendung: Android App
		\item IDE: Android Studio
		\item Programmiersprache: Java
		\item Layout: XML
		\item Persistenz: JSON
		\item Versionierung: GitLab/GitHub
	\end{itemize}
\end{frame}

\subsection{Herausforderungen}
	\begin{frame}[t]{Herausforderungen}
		\begin{itemize}
			\item für uns Neue Umgebung
			\item neue Zielplattform und damit neue Art z.B. Bilder einzubinden
		\end{itemize}
	\end{frame}
\section{weitere Möglichkeiten}
\begin{frame}[t]{weitere Möglichkeiten}
\begin{itemize}
	\item fachlich
	\begin{itemize}
		\item fachgerechte Videos und entsprechende Fragen bzw. Antworten
		\item wo das Kind z.B. eine To-Do Liste abrufen kann, was es machen soll,
		wenn z.B. der Bus nicht kommt oder welchen Weg muss ich von der Schule bis nach
		Hause gehen?
		\item Shop, in dem man Erweiterungen gegen Sterne eintauschen kann
	\end{itemize}
	\item technisch
		\begin{itemize}
		\item Stabilität verbessern $\rightarrow$ Fehlerfangen, Bugs beseitigen
		\item auch für andere OS programmieren 
	\end{itemize}
	\item grafisch
	\begin{itemize}
		\item Einbau des Begleiters als Figur
		\item Überarbeitung der auf der Oberfläche befindlichen Bilder und des Pfades
		\item Animieren z.B. des Begleiters
		\item anregendes Aussehen um Interesse aufrecht zu erhalten
	\end{itemize}
	
\end{itemize}

\end{frame}

\section{Live Demonstration}
\begin{frame}[t]{Demo}
	\begin{center}
		\begin{Huge}
			Live Demonstration
		\end{Huge}
	\end{center}
\end{frame}

\section{Quellen}
\begin{frame}{Quellen}
	\begin{columns}
		\begin{column}{16cm}	
			\begin{itemize}
				\item 
			\end{itemize}
		\end{column}
	\end{columns}
\end{frame}
\end{document}


